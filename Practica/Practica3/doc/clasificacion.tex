%!TeX spellcheck = es-ES
\chapter{Problema de clasificación}
\section{Definición del problema}
El conjunto que utilizamos en este problema es: `Optical Recognition of Handwritten Digits'.\\
Se trata de clasificar una serie de imágenes de números escritos a mano, según pertenezcan a los dígitos del 0 al 9. Estas imágenes se representan como vectores de 65 datos, siendo los 64 primeros enteros en el intervalo $[0,16]$ y el último la clase a la que pertenece el elemento $x$ (dígitos del 0 al 9).\\
Por tanto se trata de un problema de clasificación en el que cada dato $x_i \in X$ puede pertenecer a 1 de 10 clases distintas y por esto, la solución al problema sera un conjunto de 10 hiperplanos separadores, en el que cada hiperplano dividirá el espacio de puntos de tal manera que a un lado del hiperplano $w_i$ estaran los puntos $x_i \in C_i$ --- siendo $C_i$ la clase del 0 al 9 --- y al otro lado los puntos $ x_j \notin C_i  $.\\
Esta estrategia de clasificación multiclase se conoce como `one vs the all' --- OVA a partir de ahora ---.

\section{Clases de funciones}
La clase de funciones que utilizaremos para el ajuste de los datos es la clase de funciones lineales.

\section{Conjuntos de training y test}
Para este conjunto de datos ya se encuentran creados los conjuntos de training y test\cite{latexcompanion}


\section{Preprocesado de datos}

\section{Métrica del error}

\section{Técnicas de ajuste}

\section{Regularización}

\section{Modelos utilizados}

\section{Estimación de hiperparámetros}

\section{Selección del mejor modelo}

\section{Estimación por validación cruzada}

\section{Conclusiones}
